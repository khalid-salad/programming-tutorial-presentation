\section{Introduction}
\begin{frame}{Things to Consider}
    Good code depends on
    \pause
    \begin{itemize}[<+->]
        \item Correctness
        \item Readability
        \item Performance
    \end{itemize}
\end{frame}
\subsection{Correctness}
\begin{frame}{Correctness}
    \pause
    \begin{itemize}[<+->]
        \item Code should be correct
        \item If code is incorrect then doesn't matter
              \begin{itemize}[<+->]
                  \item how performant it is
                  \item how readable it is
              \end{itemize}
    \end{itemize}
    \pause
    \inputpython[Boolean Satisfiability]{code/incorrect.py}
    \pause
    \begin{itemize}[<+->]
        \item Above code is $\bigO{1}$ solution to an NP-Complete Problem
        \item But obviously not a correct solution
    \end{itemize}
\end{frame}
\begin{frame}{Testing}
    \pause
    \begin{itemize}[<+->]
        \item You must test your code
        \item Easiest way to catch bugs
        \item Example
    \end{itemize}
    \pause
    \inputpython[Primality Checker]{code/incorrect-is-prime.py}
    \pause
    \begin{itemize}[<+->]
        \item What is wrong with above code?
        \item Might not be obvious, but easy to see with testing
    \end{itemize}
\end{frame}
\begin{frame}
    \inputpython[Test \mintinline{python}|is_prime|]{code/is-prime-test.py}
    \pause
    \begin{minipage}{0.4\textwidth}
        \begin{tabular}{lcc}
            $i$ & \mintinline{python}|is_prime(i)| & Correct?   \\\toprule
            2   & \PyTrue                          & \checkmark \\
            3   & \PyTrue                          & \checkmark \\
            4   & \PyTrue                          & \xmark     \\
            5   & \PyTrue                          & \checkmark \\
            6   & \PyTrue                          & \xmark     \\
            7   & \PyTrue                          & \checkmark \\
            8   & \PyTrue                          & \xmark     \\
            9   & \PyTrue                          & \xmark     \\
            10  & \PyFalse                         & \checkmark \\\bottomrule
        \end{tabular}\end{minipage}
    \pause
    \begin{minipage}{0.55\textwidth}
        \begin{itemize}[<+->]
            \item Incorrect for composite values 4, 6, 8, and 9
            \item For these values $\mintinline{python}|int(n ** 0.5)| = \floor{\sqrt{n}} = 2$
            \item Off-by-one error -- add 1 to range: \mintinline{python}|int(n ** 0.5) + 1|
        \end{itemize}\end{minipage}
\end{frame}

\begin{frame}{Edge Cases}
    \pause
    \begin{itemize}[<+->]
        \item Be sure to test for \alert{edge cases}
              \begin{itemize}[<+->]
                  \item boundary or degenerate cases, e.g.,
                        \begin{itemize}[<+->]
                            \item sorting an \alert{empty} list
                            \item binary search returning \alert{first} or \alert{last} index
                            \item checking if 0 or 1 is \alert{prime}
                            \item traversing a graph with \alert{no edges}
                        \end{itemize}
              \end{itemize}
        \item Depending on the context, you may also need to test for \alert{invalid inputs}.
    \end{itemize}
\end{frame}

\subsection{Readability}
\begin{frame}{Readability}
    \pause
    \begin{itemize}[<+->]
        \item Usually the most important thing after correctness
        \item Code will be read far more often than it is written
        \item Maintenance of code is often the highest expense
    \end{itemize}
\end{frame}

\begin{frame}{Syntax}
    \begin{itemize}[<+->]
        \item Follow a set of best-practices for your language. For example
              \begin{itemize}[<+->]
                  \item In Python, there is PEP 8
                  \item In C, there is the Linux Kernel Style
              \end{itemize}
        \item The style chosen is less important than that you follow a consistent style
    \end{itemize}
    \pause
    \begin{multicols}{2}
        \inputpython[Ugly]{code/unreadable.py}
        \pause
        \inputpython[Readable]{code/readable.py}
    \end{multicols}
    \pause
    \begin{itemize}[<+->]
        \item Easiest way to be consistent is to use a formatter
        \item Automatically formats your code according to some style guide
              \begin{itemize}[<+->]
                  \item Black, yapf, autopep8, etc., for Python
                  \item clang-format for C/C++
              \end{itemize}
    \end{itemize}
\end{frame}

\begin{frame}{Structure}
    \begin{itemize}[<+->]
        \item Syntactic readability (style) is not enough
        \item Code should be structured for readability
        \item \alert{Modular} and \alert{structured} code is easier to
              \begin{itemize}[<+->]
                  \item read
                  \item write
                  \item debug
              \end{itemize}
        \item Consider following problem
    \end{itemize}
    \pause
    \begin{block}{Problem}
        Find the largest product of two 3-digit numbers that is a palindrome.
    \end{block}
    \pause
    \begin{block}{Brute-Force Solution}
        Iterate through all $10^6$ pairs of products and check if they are palindromic while keeping track of max.
    \end{block}
\end{frame}
\begin{frame}{Example of Bad Structure}
    \pause
    \resizebox*{0.6\textwidth}{!}{\inputpython[Poorly Structured]{code/unstructured.py}}
    \pause
    \begin{minipage}{0.3\textwidth}
        \vspace{-2.5in}\begin{itemize}[<+->]
            \item code is hard to follow
            \item unnecessarily complex --- \mintinline{python}|is_palindrome| is essentially a sentinel value
            \item code is imperative, not declarative
        \end{itemize}
    \end{minipage}
\end{frame}

\begin{frame}{Example of Better Structure}
    \pause
    \resizebox*{0.6\textwidth}{!}{\inputpython[Better Structured]{code/structured.py}}
    \pause
    \begin{minipage}{0.3\textwidth}
        \vspace{-2in}\begin{itemize}[<+->]
            \item code is much easier to follow
            \item function \mintinline{python}|is_palindrome| clearly conveys intent
            \item code is declarative
        \end{itemize}
    \end{minipage}
\end{frame}

\begin{frame}{Example of Good Structure}
    \pause
    \resizebox*{0.6\textwidth}{!}{\inputpython[Well Structured]{code/good-structure.py}}
    \pause
    \begin{minipage}{0.3\textwidth}
        \vspace{-2in}\begin{itemize}[<+->]
            \item basically same code
            \item clearly describes what the program is doing \alert{overall}
        \end{itemize}
    \end{minipage}
\end{frame}

\begin{frame}{Comments}
    \pause
    \begin{itemize}[<+->]
        \item Code should be commented
        \item but not \alert{over commented}
    \end{itemize}
    \pause
    \inputpython[Bad Comment]{code/bad-comment.py}
    \pause
    \begin{itemize}[<+->]
        \item it is clear that \mintinline{python}|i += 1| increments \mintinline{python}|i|
        \item comment is superfluous and \alert{distracting}
    \end{itemize}
\end{frame}

\begin{frame}{Good Comments}
    \pause
    \begin{itemize}[<+->]
        \item As a rule of thumb
              \begin{itemize}[<+->]
                  \item good \alert{code} describes \alert{what} and \alert{how}
                  \item good \alert{comments} describe \alert{why}
              \end{itemize}
    \end{itemize}
    \pause
    \resizebox{0.6\textwidth}{!}{\inputpython[Good Comment]{code/good-comment.py}}
    \pause
    \begin{minipage}{0.35\textwidth}\vspace*{-1.75in}\begin{itemize}[<+->]
            \item this comment explains \alert{why} the code is iterating from \mintinline{python}|1| to \mintinline{python}|int(n ** 0.5) + 1|
            \item without this comment, reader would have to determine for themselves
        \end{itemize}\end{minipage}
\end{frame}

\begin{frame}{Good Code}
    \begin{itemize}[<+->]
        \item If your code is well-written, it will improve readability
        \item Not just in terms of structure, but things like variable and function names
        \item Well-written code often makes many comments \alert{unnecessary}
        \item Often expressed as
              \begin{block}{}
                  Good Code is its Own Best Documentation
              \end{block}
        \item Not an excuse to avoid comments
    \end{itemize}
\end{frame}

\begin{frame}{Good Naming}
    \onslide<2-11>{\resizebox{0.45\textwidth}{!}{\inputpython[Bad Names]{code/bad-name.py}}}
    \onslide<5-11>{\resizebox{0.45\textwidth}{!}{\inputpython[Good Names]{code/good-name.py}}}
    \pause
    \begin{multicols}{2}
        \begin{itemize}
            \item<3-4> these names convey
                  \begin{itemize}[<+->]
                      \item<4> basically nothing
                  \end{itemize}
        \end{itemize}
        \vfill\null
        \columnbreak
        \begin{itemize}
            \item<6-10> these names convey
                  \begin{itemize}
                      \item<7-10> function is quicksort
                      \item<8-10> input is an array
                      \item<9-10> \mintinline{python}|left| and \mintinline{python}|right| are partitions around \mintinline{python}|pivot = array[0]|
                      \item<10> \mintinline{python}|ele| iterates through values of array
                  \end{itemize}
        \end{itemize}
    \end{multicols}
\end{frame}

\begin{frame}{Miscellaneous}
    \pause
    \begin{itemize}[<+->]
        \item Impractical to create exhaustive list of best practices
              \begin{itemize}[<+->]
                  \item and some practices are debateable
              \end{itemize}
        \item Good idea to look at resources like
              \begin{itemize}[<+->]
                  \item The Little Book of Python Anti-Patterns
                  \item The C++ Core Guidelines
              \end{itemize}
        \item And to use a \alert{linter} --- a static code analysis tool to flag bugs, style errors, etc, such as
              \begin{itemize}[<+->]
                  \item Python -- flake8
                  \item C++ -- clang-tidy
              \end{itemize}
    \end{itemize}
\end{frame}

\subsection{Performance}
\begin{frame}{Performance}
    \pause
    \begin{itemize}[<+->]
        \item Highly case-by-case
              \begin{itemize}[<+->]
                  \item specific use case may allow for less performant code
                        \begin{itemize}[<+->]\item e.g. some one-time scientific calculation\end{itemize}
                  \item or may require more performant code
                        \begin{itemize}[<+->]\item e.g. real-time financial analysis\end{itemize}
              \end{itemize}
        \item Usually comes down to correct choice of algorithm and data structure
              \begin{itemize}[<+->]
                  \item e.g., if checking existence of an element, an array gives $\bigO{n}$ but a hash table gives $\bigO{1}$
                        \begin{itemize}[<+->]\item which might make the difference between $\bigO{n^2}$ and $\bigO{n}$ overall\end{itemize}
              \end{itemize}
        \item While it does depend on use case, it is still usually best to focus on readability over performance
              \begin{itemize}[<+->]\item it's easier to make slow code fast than to make confusing code understandable\end{itemize}
    \end{itemize}
\end{frame}